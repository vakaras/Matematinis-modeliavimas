\Chapter{Pavyzdžiai}

\section{Sintaksės paryškinimas}

Python fragmentas:
\begin{minted}{python}
def boring(args=None):
  pass
\end{minted}

Taip pat vienos eilutės fragmentas: \mint{python}|import this|.

Scala fragmentas:
\begin{minted}{scala}
object TimerAnonymous {
  def oncePerSecond(callback: () => Unit) {
    while (true) {
      callback()
      Thread sleep 1000
      }
    }
  def main(args: Array[String]) {
    oncePerSecond(
      () => println("Time flies like an arrow…"))
    }
  }
\end{minted}

\section{Apibrėžimai, lemos, teoremos, pavyzdžiai…}

\subsection{Teiginio su įrodymu pavyzdys}

\begin{prop}
  Jei $\lim_{x \to a} f(x) = b$ ir $\lim_{x \to a} f(x) = c$, tai $b = c$.
  \begin{proof}
    Tarkime priešingai $b \neq c$. Tegu $b < c$.

    Pagal \ref{limfed} apibrėžimą:
    \begin{align*}
      \lim_{x \to a} f(x) = b \iff &
        \forall U_{b}, \exists U_{a} :
        f(x) \in U_{b}, \forall x(x \in U_{a}) \\
      \lim_{x \to a} f(x) = c \iff &
        \forall U_{c}, \exists U'_{a} :
        f(x) \in U_{c}, \forall x(x \in U'_{a})
    \end{align*}

    Pastebėkime, kad $U_{a} \cap U'_{a}$ bus taško $a$ aplinka (taško
    aplinka visada yra netuščia aibė).
    Fiksuojame tokias $U_{b}$ ir $U_{c}$, kurioms
    $U_{b} \cap U_{c} = \emptyset$. Bet
    $\forall x (x \in U_{a} \cap U'_{a}) \implies
      f(x) \in U_{b} \land f(x) \in U_{c}$, gavome prieštarą.
  \end{proof}
\end{prop}

\subsection{Apibrėžimų pavyzdžiai}

\begin{defn}[Funkcijos riba]
  \label{limfed}
  Taškas $b$ vadinamas funkcijos $f$ riba taške $a$, jei:
  \begin{equation*}
    \forall U_b, \exists U_a : f(x) \in U_b,
    \forall x (x \in U_a \cap A \setminus \{a\})
  \end{equation*}
\end{defn}

\begin{defn}[\XeTeX]
  \TeX teksto rinkimo variklis \en{\TeX{} typesetting engine} naudojantis
  Unikodą ir palaikantis modernias šriftų technologijas, tokias kaip
  OpenType ir AAT.
\end{defn}

\subsection{Pavyzdžių pavyzdžiai}

\begin{exmp}
  Raskime $\sin x^{2}$ išvestinę.

  Pažymėkime
  \begin{align*}
    f(x) &= \sin x \\
    g(x) &= x^{2}. \\
    \intertext{Tada}
    (f \circ g)(x) &= \sin x^{2}.\\
    \intertext{Dabar galime pritaikyti išvestinės skaičiavimo formulę:}
    g'(x) &= (x^{2})' = 2x \\
    f'(x) &= (\sin x)' = \cos x \\
    (f \circ g)'(a) &= (\cos x)|_{x = a^{2}} \cdot (2x)|_{x=a}
      = (\cos a^2)(2a). \\
    \intertext{Taigi}
    (\sin x^{2})' &= 2x \cos x^{2}
  \end{align*}

\end{exmp}

\subsection{Pastabų pavyzdžiai}

\begin{note}
  $f : A \to \RSET, A \subset \RSET, a(a \in A)$ yra vidinis aibės
  $A$ taškas.
\end{note}

\subsection{Žymėjimų pavyzdžiai}

\begin{notation}
  \begin{align*}
    f'(a)
    &\equiv \frac{d f(a)}{dx} \\
    &\equiv \left. \frac{d f(x)}{dx} \right|_{x = a} \\
    &\equiv \left. f'(x) \right|_{x = a}.
  \end{align*}
\end{notation}

\section{Lentelės}

\xtable
{
  w [ 3 | 5 ]
  a [ p | p ]
  h [ Foo | Bar ]
  %
  e [ Čia yra nenumeruotas sąrašas: |
  @begin{itemize}
    @item Kažkas.
    @item Kažkas kitas.
  @end{itemize}
  ]
  e [O čia yra @strong{matematikos} formulė: |
  $@lim _{n @to @infty} @frac{1}{n} = 0$
  ]
  e [ O čia citata: |
  @cite&L12&R{bdr97}
  ]
  e [ O čia MIF'iečio elektroninio pašto adresas: |
  vardas.pavardė&Amif.stud.vu.lt
  ]
}

\section{Nuorodos į aprašymus}

Pagal \url{http://texblog.org/category/latex/}.

\begin{description}[style=multiline, labelwidth=2.0cm]
	\item[\namedlabel{itm:rule1}{1 taisyklė}] Su \LaTeX viskas lengva.
	\item[\namedlabel{itm:rule2}{2 taisyklė}] Kartais tai nėra taip lengva\\
		$\to$ \ref{itm:rule1}
\end{description}
