\Chapter{Tiesinės pirmos eilės lygtys}

\begin{equation}
  y' + a(x) y = f(x)
  \label{eq:sk2:1}
\end{equation}

\section{Konstantų varijavimo metodas}

Tarkime, kad $y_{1}$ ir $y_{2}$ yra \ref{eq:sk2:1} lygties sprendiniai.
Jei $y = y_{1} - y_{2}$, tai įsistatę sprendinius į \ref{eq:sk2:1},
gausime:
\begin{equation*}
  -\left\{%
  \begin{aligned}
    y'_{1} + a(x) y_{1} = f(x) \\
    y'_{2} + a(x) y_{2} = f(x) \\
  \end{aligned}%
  \right.
\end{equation*}
\begin{equation*}
  y' + a(x) y = 0.
\end{equation*}

Taigi bendrasis \ref{eq:sk2:1} lygties sprendinys yra:
\begin{equation}
  y = y_{h} + y_{a},
  \label{eq:sk2:2}
\end{equation}
čia:
\begin{description}
  \item[$y_{h}$] – homogeninis lygties sprendinys;
  \item[$y_{a}$] – atskiras lygties sprendinys.
\end{description}

$y_{h}$ randame spręsdami homogeninę lygtį:
\begin{align*}
  y' + a(x) y &= 0 \\
  y' &= \frac{dy}{dx} \\
  \frac{dy}{dx} + a(x) y &= 0 \\
  \frac{dy}{dx} &= - a(x) y \\
  \frac{dy}{y} &= - a(x) dx, y \neq 0 \\
  \ln |y| &= - \int a(x) dx + \ln C \\
\end{align*}
Taigi homogeninis lygties sprendinys yra:
\begin{equation}
  y_{h} = C e^{-\int a(x) dx}. \\
  \label{eq:sk2:3}
\end{equation}
Pastaba: $y=0$ irgi yra sprendinys, bet jis jau priklauso \ref{eq:sk2:3}.

Ieškome atskirojo sprendinio $y_{a}$:
\begin{equation*}
  y_{a} = c(x) e^{-\int _{x_{0}}^{x} a(s) ds}.
\end{equation*}
Jo išvestinė:
\begin{equation*}
  y'_{a} = c'(x) e^{-\int _{x_{0}}^{x} a(s) ds} +%
    c(x) e^{-\int _{x_{0}}^{x} a(s) ds}\cdot(-a(x))
\end{equation*}
Įsistatę į \ref{eq:sk2:1} gauname:
\begin{equation*}
  c'(x) e^{-\int _{x_{0}}^{x} a(s) ds} +%
    c(x) e^{-\int _{x_{0}}^{x} a(s) ds}\cdot(-a(x)) +%
  a(x) \cdot c(x) e^{-\int _{x_{0}}^{x} a(s) ds} = f(x)
\end{equation*}
Suprastiname ir suintegruojame:
\begin{align*}
  c'(x) e^{-\int _{x_{0}}^{x} a(s) ds} &= f(x) \\
  c'(x) &= f(x) e^{\int _{x_{0}}^{x} a(s) ds} \\
  c(x) &= \int _{x_{0}} ^{x} f(t) e^{\int _{x_{0}}^{t} a(s) ds} dt + C_{1}
\end{align*}
Kadangi $C_{1}$ galim pasirinkti bet kokį, tai renkamės 0. Taigi
atskirasis sprendinys yra:
\begin{equation*}
  y_{a} =
    e^{-\int _{x_{0}}^{x} a(s) ds} \cdot 
    \int _{x_{0}} ^{x} f(t) e^{\int _{x_{0}}^{t} a(s) ds} dt
\end{equation*}

O bendrasis sprendinys:
\begin{align*}
  y
  &= y_{h} + y_{a} \\
  &= C e^{-\int a(x) dx} + 
    e^{-\int _{x_{0}}^{x} a(s) ds} \cdot 
    \int _{x_{0}} ^{x} f(t) e^{\int _{x_{0}}^{t} a(s) ds} dt
\end{align*}

Jei turime papildomą sąlygą $y(x_{0}) = y_{0}$, tai
\begin{align*}
  y(x_{0}) &= C \cdot 1 + 1 \cdot 0 \\
  C &= y_{0} \\
\end{align*}

\begin{align*}
  y
  &= y_{0} e^{-\int a(x) dx} + 
    e^{-\int _{x_{0}}^{x} a(s) ds} \cdot 
    \int _{x_{0}} ^{x} f(t) e^{\int _{x_{0}}^{t} a(s) ds} dt
\end{align*}

\section{Bernulio metodas}

