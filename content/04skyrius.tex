\Chapter{Diferencialinių lygčių sistemos}

\section{Sąvokos}

\cite[109]{konspektas}

\begin{defn}[Itegralinė kreivė]
  Tegu $y = \varphi(x)$, $x \in (a, b)$ yra diferencialinių lygčių
  sistemos sprendinys. Tada funkcija $\varphi$ srityje
  $G (G \subset \RSET^{n+1})$ apibrėžia kreivę
  \begin{equation*}
    l = \left\{ (x, y) \in \RSET^{n+1} :%
      y = \varphi(x), x \in (a, b) \right\} \subset \RSET^{n+1},
  \end{equation*}
  kuri yra vadinama šios sistemos integraline kreive.
\end{defn}

\begin{defn}[Fazinė trajektorija]
  \begin{equation*}
    \left\{ y \in \RSET^{n} : y = \varphi(x), x \in (a, bj) \right\}%
      \subset \RSET^{n}
  \end{equation*}
  Taip apibrėžta kreivė, kartu su apėjimo kryptimi, yra vadinama
  fazine trajektorija, o erdvė $\RSET^{n}$ – fazine erdve.
  Kitaip tariant fazinė trajektorija yra integralinės kreivės
  projekcija lygiagrečiai $x$ ašiai.
\end{defn}

\begin{defn}[Pusiausvyros taškas]
  Fazinė trajektorija sutampanti su tašku yra vadinama pusiausvyros
  tašku (kartais ramybės tašku). Taškas $y$ yra pusiausvyros
  taškas tada ir tik tada, kai:
  \begin{equation*}
    f(x, y) = 0, \forall x (x \in (a, b)).
  \end{equation*}
\end{defn}

\section{Tiesinės homogeninių diferencialinių lygčių sistemos}

Tiesinė homogeninė lygčių sistema:
\begin{equation}
  y' = A(x) y.
  \label{eq:sk4:1}
\end{equation}

\begin{prop}
  Jeigu $\varphi(x)$ yra \ref{eq:sk4:1} lygties sprendinys, tai ir
  $C\varphi(x)$ yra sprendinys.
  \begin{proof}
    \begin{equation*}
      (C\varphi(x))' = C\varphi'(x) = CA(x)\varphi(x) = A(x)(C\varphi(x))
    \end{equation*}
  \end{proof}
\end{prop}

\begin{prop}
  Jeigu $\varphi(x)$ ir $\psi(x)$ yra sprendiniai, tai ir
  $\varphi(x) + \psi(x)$ irgi yra sprendinys.
  \begin{proof}
    \begin{equation*}
      (\varphi(x) + \psi(x))' =%
      \varphi'(x) + \psi'(x) =%
      A(x)\varphi(x) + A(x)\psi(x) =%
      A(x)(\varphi(x) + \psi(x))
    \end{equation*} 
  \end{proof}
\end{prop}

Išvados:
\begin{enumerate}
  \item Jeigu $\varphi_{1},\varphi_{2},\ldots,\varphi_{m}$ yra
    \ref{eq:sk4:1} sistemos sprendiniai, tai ir
    $C_{1}\varphi_{1}, C_{2}\varphi_{2},\ldots,C_{m}\varphi_{m}$
    yra \ref{eq:sk4:1} sistemos sprendiniai.
  \item Tegu $z = \varphi + i \psi$ yra kompleksinės \ref{eq:sk4:1}
    lygčių sistemos su realiais koeficientas $a_{k,j} (k,j = 1..n)$
    sprendinys. Tada $\varphi$ ir $\psi$ irgi yra sprendiniai.
\end{enumerate}

\begin{defn}
  Vektorinės funkcijos
  $\varphi_{1}, \varphi_{2}, \ldots, \varphi_{m} : (a, b) \to \RSET^{n}$
  yra tiesiškai nepriklausomos tada ir tik tada, kai iš to, kad
  \begin{equation*}
    c_{1}\varphi_{1}(x) + \cdots + c_{m}\varphi_{m}(x) = 0,%
    \forall x (x \in (a, b))
  \end{equation*}
  išplaukia, kad
  \begin{equation*}
    c_{1} = c_{2} = \cdots = c_{m} = 0.
  \end{equation*}
\end{defn}

\begin{prop}
  Bet kokie $n$ tiesiškai nepriklausomi \ref{eq:sk4:1} sistemos
  sprendiniai sudaro šios sistemos sprendinių bazę. Kitaip tariant
  per juos galima išreikšti visus kitus sistemos sprendinius.
\end{prop}

\begin{defn}[Funadamentaliųjų sprendinių matrica]
  Tarkime $\varphi_{1}, \varphi_{2}, \ldots, \varphi_{n}$ yra
  $n$ tiesiškai nepriklausomi \ref{eq:sk4:1} sistemos sprendiniai
  (sprendinių bazė). Juos vadinsime fundamentaliąja sprendinių
  sistema, o iš jų sudarytą matricą:
  \begin{equation*}
    \Phi = \left( \varphi_{1}, \varphi_{2}, \ldots, \varphi_{n} \right)
  \end{equation*}
  fundamentaliąja sprendinių matrica.
\end{defn}

\begin{defn}[Vronskio determinantas]
  Fundamentaliųjų sprendinių matricos $\Phi$ determinantas yra
  vadinamas funkcijų sistemos $\varphi_{1}, \ldots, \varphi_{n}$
  Vronskio determinantu:
  \begin{equation*}
    W(x) = |\Phi(x)|.
  \end{equation*}
\end{defn}

\begin{prop}
  Yra ekvivalentūs tokie trys teiginiai:
  \begin{enumerate}
    \item $W(x) = 0, \forall x (x \in (a, b))$;
    \item $W(x_{0}) = 0$, kokiame nors taške $x_{0} \in (a, b)$;
    \item Sprendiniai $\varphi_{1}, \ldots, \varphi_{n}$ – tiesiškai
      priklausomi.
  \end{enumerate}
\end{prop}

\begin{defn}[Liuvilio formulė]
  \begin{equation*}
    W(x) = W(x_{0})e^{\int _{x_{0}}^{x} \sum _{k=1}^{n}a_{kk}(x)dx}
  \end{equation*}
\end{defn}

\section{Nehomogeninės tiesinių diferencialinių lygčių sistemos}

\begin{equation}
  y' = A(x) y + q(x)
  \label{eq:sk4:2}
\end{equation}

Tarkime, kad vektorius $y = \psi(x), x \in (a, b)$ yra
\ref{eq:sk4:2} sistemos sprendinys. Padarę keitinį $y = \varphi + \psi$,
($\varphi$ yra kažkokia nauja funkcija) gausime tiesinę homogeninę
lygčių sistemą:
\begin{align*}
  y' &= A(x) y + q(x) \\
  (\varphi + \psi)' &= A(x)(\varphi + \psi) + q(x) \\
  \varphi' + \psi' &= A(x)\varphi + A(x)\psi + q(x) \\
\end{align*}
\begin{equation}
  \varphi' = A(x)\varphi.
  \label{eq:sk4:3}
\end{equation}

Taigi \ref{eq:sk4:2} bendrasis sprendinys:
\begin{equation*}
  y = y_{a} + y_{h},
\end{equation*}
čia $y_{a}$ – bet koks sprendinys, o $y_{h}$ – homogeninis sprendinys.

Tegu $\varphi_{1}, \varphi_{2}, \ldots, \varphi_{n}$ yra
fundamentalioji \ref{eq:sk4:3} sistemos sprendinių sistema, o
$\Phi$ – iš jų sudaryta fundamentalioji matrica. Atskirą sprendinį
rasime konstantų varijavimo metodu.

Apibrėžkime funkciją:
\begin{equation*}
  \gamma(x) = \Phi(x)c(x),
\end{equation*}
čia $c(x) : (a, b) \to \RSET^{n}$ – nežinoma vektorinė funkcija.

\TODO{Suprasti iki \cite[118]{konspektas}.}

\section{Tiesinių diferencialinių lygčių sistemos su pastoviais
realiais koeficientais}

\cite[119]{konspektas}

Idėja yra ieškoti sprendinio, kurio forma:
\begin{equation*}
  y = be^{\lambda x},
\end{equation*}
kur $y$ ir $b$ yra vektoriai, o $y$ – skaliaras. Šį sprendinį įsistačius
į pradinę lygtį yra gaunama charakteristinė lygtis:
\begin{align*}
  (b e^{\lambda x})' &= A b e^{\lambda x} \\
  \lambda b e^{\lambda x} &= A b e^{\lambda x} \\
  (E \lambda - A) b e^{\lambda x} &= 0 \\
  (E \lambda - A) &= 0 \\
\end{align*}

\subsection{Pastaba}

%Iš $n$ nepriklausomų sprendinių galime sudaryti fundamentalią
%matricą: $\Phi = (\varphi_{1}, \ldots, \varphi_{n})$.

Tegu $A$ yra $n \times n$ eilės matrica su realiais koeficientais.
Tiesinę sistemą $y' = Ay$, panaudoję keitinį $y = Qz$, kur
$|Q| \neq 0$ galime suvesti:
\begin{align*}
  y' &= Ay \\
  Qz' &= AQz \\
  z' &= Q^{-1}AQz \\
\end{align*}
$Q$ galime parinkti taip, kad $Q^{-1}AQ = J$. Čia $J$ – Žordano
matrica. Kartu tiesinę sistema su pastoviais koeficientais
galima suvesti į paprastesnę sistemą:
\begin{equation*}
  z' = Jz.
\end{equation*}
Ši sistema vadinama kanonine.

\section{Kanoninių sistemų plokštumoje faziniai portretai}

\cite[126-129]{konspektas}
