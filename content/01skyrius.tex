\Chapter{Pirmos eilės diferencialinės lygtys}

Diferencialinė lygtis neišreikšta išvestinės atžvilgiu:
\begin{equation}
  F (x, y, y') = 0.
  \label{eq:sk1:1}
\end{equation}

Diferencialinė lygtis išreikšta išvestinės atžvilgiu:
\begin{equation}
  y' = f(x, y).
  \label{eq:sk1:2}
\end{equation}

\begin{defn}[Integralinė kreivė]
  Tarkime, kad $y = \varphi(x), x \in (a, b)$ yra \ref{eq:sk1:2} lygties
  sprendinys. Tada ši lygtis ($y = \varphi(x)$) apibrėžia kreivę
  $l$, kuri vadinama integraline kreive.
\end{defn}

\begin{defn}[Krypčių laukas]
  Kiekvienam taškui galime priskirti atkarpą su krypties koeficientu
  $k = f(x, y)$. Tokių atkarpėlių visuma vadinama krypčių lauku.
\end{defn}

\begin{defn}[Izoklina]
  Kreivė, kurioje krypčių lauko atkarpėlių kryptys yra vienodos.
  Ji yra apibrėžiama lygtimi:
  \begin{equation*}
    c = f(x, y),
  \end{equation*}
  čia $c$ – yra pasirinkta kryptis.
\end{defn}

Integralinės lygties bendrasis sprendinys:
\begin{equation}
  y = \varphi(x, C).
  \label{eq:sk1:3}
\end{equation}

Integralinės lygties bendrasis integralas:
\begin{equation}
  \Phi(x, y, C) = 0.
  \label{eq:sk1:4}
\end{equation}

\begin{defn}[Bendrasis sprendinys]
  Tegu $\varphi \in D, D \subset \RSET^{2}$. Sakysime, kad lygtis
  $y = \varphi(x, C)$ $(x, C \in D)$ apibrėžia bendrąjį diferencialinės
  lygties $y' = f(x, y), (x, y \in G)$ sprendinį, jei:
  \begin{enumerate}
    \item $\forall (x_{0}, y_{0}) \in G, y_{0} = \varphi(x_{0}, C) \implies%
      \exists C_{0}, C_{0} = C(x_{0}, y_{0})$
    \item $(x_{0}, y_{0}) \in D$ ir $y = \varphi(x, C_{0})$ yra Koši
      uždavinio $y' = f(x, y), (x, y) \in G, y(x_{0}) = y_{0}$
      sprendinys.
  \end{enumerate}
\end{defn}

\begin{defn}[Atskirasis sprendinys]
  Sprendinį $y = \varphi(x, C_{0})$, kurį gauname iš bendrojo sprendinio
  $C$ pakeitę į $C_{0}$, vadiname atskiruoju sprendiniu.
\end{defn}

\begin{defn}[Įpatingasis sprendinys]
  Sprendinį $y = \varphi(x)$ mes vadinsime ypatinguoju, jei per
  kiekvieną jo tašką eina dar bent viena diferencialinės lygties
  integralinė kreivė.
\end{defn}

\section{Diferencialinių lygčių sprendimo metodai}

\subsection{Lygtys su atskiriamais kintamaisiais}

\subsection{Lygtys, kurias įmanoma suvesti į lygtį su atskirtaisiais
kintamaisiais}

\subsubsection{Polinomas}

\begin{equation*}
  y' = f(ax + by + c), a, b, c \in \RSET
\end{equation*}

Keitinys:
\begin{align*}
  v &= ax + by + c \\
  v' &= a + by' \\
\end{align*}

\subsubsection{Homogeninė nulinio laipsnio funkcija}

\begin{align*}
  f(\lambda x, \lambda y) &= \lambda^{0} f(x, y) \\
  y' &= f(x, y) \\
  y' &= f(\lambda x, \lambda y) \\
  \intertext{Pakeičiam $\lambda = \frac{1}{x}$:}
  &= f(1, \frac{y}{x}) \\
\end{align*}

\subsubsection{Polinomų dalyba}

\begin{equation*}
  y' = f \left( \frac{ax + by + c}{mx + ny + d} \right),
  a, b, c, m, n, d \in \RSET
\end{equation*}

Iš lygčių sistemos:
\begin{equation*}
  \left\{%
  \begin{aligned}
    ax + by + c &= 0 \\
    mx + ny + d &= 0 \\
  \end{aligned}%
  \right.
\end{equation*}
gauname sprendinį $(x_{0}, y_{0})$ ir atliekame keitinį:
\begin{align*}
  u &= x - x_{0} \\
  v &= y - y_{0} \\
\end{align*}
