\Chapter{Autonominės sistemos}

\section{Autonominės lygtys tiesėje}

\cite[133]{konspektas}

Geometrinis sprendinių kokybinis vaizdas $x$ ašyje vadinamas faziniu
portretu, $x$ ašis – fazine ašimi, o jos taškai – faziniais taškais.

\section{Autonominės sistemos plokštumoje}

\cite[137]{konspektas}

\cite[147]{konspektas}

\begin{prop}
  Autonominės sistemos trajektorijos gali būti tik tokių trijų
  rūšių:
  \begin{enumerate}
    \item pusiausvyros taškas;
    \item uždara trajektorija, kurią atitinka $\omega$-periodinis
      sprendinys;
    \item nekertanti savęs trajektorija, kurią atitinka neperiodinis
      sprendinys.
  \end{enumerate}
\end{prop}

\begin{prop}
  Tarkime, srityje $\Omega \subset \RSET^{2}$ funkcija $f$ tenkina
  kurią nors vieną iš šių sąlygų:
  \begin{enumerate}
    \item vektorinis laukas $f$ yra potencialus srityje $\Omega$;
    \item vektorinio lauko divergencija $\div f$ turi pastovų
      ženklą srityje $\Omega$.
  \end{enumerate}
  Tada ši autonominė sistema srityje $\Omega$ neturi uždarų
  trajektorijų:
  \begin{equation*}
    x' = f(x);
  \end{equation*}
  čia matricinė šios lygčių sistemos išraiška:
  \begin{equation*}
    \left\{%
    \begin{aligned}
      x'_{1} &= f_{1}(x) \\
      x'_{2} &= f_{2}(x) \\
    \end{aligned}.
    \right.
  \end{equation*}
\end{prop}
