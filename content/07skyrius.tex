\Chapter{Dalinių išvestinių lygtys}

\section{Furjė arba kintamųjų atskyrimo metodas}

TODO: \cite[168]{konspektas}

\section{Tiesinių antros eilės dalinių išvestinių klasifikacija}

\cite[157]{konspektas}

Tiesinė antros eilės dalinių išvestinių lygtis:
\begin{equation}
  \sum ^{n} _{i=1,j=1} a_{ij}(x)u_{x_{i}x_{j}} + %
  \sum _{i=1}^{n} a_{i}(x)u_{x_{i}} + a(x)u = f(x),
  \label{eq:sk7:1}
\end{equation}
čia:
\begin{itemize}
  \item $u = u(x) = u(x_{1},x_{2},\ldots,x_{n}), x \in \RSET^{n}$;
  \item $a_{ij} = a_{ji}$;
  \item $a_{ij}$ – žinomos funkcijos.
\end{itemize}

Fiksuojame tašką $x = x^{0}$ ir
\begin{equation*}
  \Lambda(x^{0}, \xi) =
  \underbrace{
    \sum^{n}_{i=1,j=1} a_{ij}(x^{0})\xi_{i}\xi_{j}
  }_{\t{Kvadratinė forma}} =
  \sum _{k=1}^{n} \lambda_{k}\gamma_{k}^{2},
\end{equation*}
čia $\xi = (\xi_{1}, \ldots, \xi_{n})$.

\begin{equation*}
  \xi_{i} = \sum_{k=1}^{n} C_{ki}\gamma_{k}, i = 1,2,\ldots,n
\end{equation*}

Paimame neišsigimusią matricą ($det \left\{ C_{ki} \right\} \neq 0$).

Kvadratinių formų incercijos dėsnis: kvadratinės formos, suvestos
į kvadratų sumą, teigiamų, neigiamų ir lygių nuliui koeficientų
$\lambda_{k}$ skaičius nepriklauso nuo suvedančios matricos $C_{ki}$.

\begin{defn}
  Sakysime \ref{eq:sk7:1} lygtis taške $x^{0}$ yra
  ($\alpha,\beta,\gamma$) tipo jeigu kvadratinėje formoje suvestoje
  į kvadratų sumą yra $\alpha$ teigiamų, $\beta$ neigiamų ir
  $\gamma$ lygių 0 koeficientų $\lambda_{k}$.
\end{defn}

Dažniausiai pasitaikantys atvejai:
\begin{enumerate}
  \item $\lambda_{k} \neq = 0, \forall k (k = 1,\ldots,n)$,
    $\lambda_{n} \gtrless 0,$
    $\lambda_{1}, \lambda_{2},\ldots,\lambda_{n-1} \lessgtr 0$
    – hiperbolinė lygtis;
  \item $\lambda_{k} \neq 0, \forall k$ ir visi vieno ženklo –
    elipsinė lygtis;
  \item $\lambda_{n} = 0, \lambda_{i} \neq 0$ ir vieno ženklo –
    parabolinė lygtis.
\end{enumerate}
