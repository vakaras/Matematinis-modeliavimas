\Chapter{Paprasčiausi matematiniai modeliai}

\section{Variacinio skaičiavimo elementai}

\subsection{Bernulio uždavinys apie brachistochronę}

\cite[18]{konspektas}

Variacinio skaičiavimo terminais uždavinys skambėtų taip:
Tegu $y: (a, b) \to \RSET^{1}$ yra diferencijuojama argumento $x$
funkcija, kurios $y(x_{1})=y_{1}$ ir $y(x_{2})=y_{2}$. Aibėje
tokių funkcijų reikia rasti tą, kuriai TODO integralas įgyja
mažiausią reikšmę.

\subsection{Mažiausio ploto radimo uždavinys}

\cite[19]{konspektas}

\subsection{Didžiausio ploto radimo uždavinys}

\cite[20]{konspektas}

\subsection{Pagrindinė variacinio skaičiavimo lema}

\begin{prop}
  Tarkime, kad:
  \begin{equation*}
    f \in C(a, b) \land \int _{a}^{b}f(x)\gamma(x)dx = 0, %
      \forall \gamma (\gamma \in C_{0}^{1}(a, b))
  \end{equation*}
  tada:
  \begin{equation*}
    f(x) \equiv 0, \forall x (x \in (a, b)).
  \end{equation*}
  Čia $C^{1}_{0}$ reiškia, kad turi tolygias išvestines ir kraštiniuose
  taškuose yra lygi 0.
\end{prop}

\cite[23]{konspektas}

\newcommand{\MSET}{\mathfrak{M}}

\TODO{Suprasti iki \cite[25]{konspektas} (įskaitant).}

\begin{defn}[Fukcijos stiprioji aplinka]
  Tegu $\varepsilon > 0$ yra fiksuotas skaičius ir $y \in \MSET$.
  Funkcijos $y$ nulinės eilės (arba stipriąja) $\varepsilon$
  aplinka vadinsime aibę
  \begin{equation*}
    \MSET_{0} = \left\{ \tilde{y} \in \MSET :%
    \max _{x\in\left[ a, b \right]} | \tilde{y}(x) - y(x) | %
    \leq \varepsilon \right\}.
  \end{equation*}
\end{defn}

\begin{defn}[Fukcijos silpnoji aplinka]
  Tegu $\varepsilon > 0$ yra fiksuotas skaičius ir $y \in \MSET$.
  Funkcijos $y$ pirmosios eilės (arba silpnąja) $\varepsilon$
  aplinka vadinsime aibę
  \begin{equation*}
    \MSET_{0} = \left\{ \tilde{y} \in \MSET :%
    \max _{x\in\left[ a, b \right]} | \tilde{y}(x) - y(x) |%
    \max _{x\in\left[ a, b \right]} | \tilde{y}'(x) - y'(x) |%
    \leq \varepsilon \right\}.
  \end{equation*}
\end{defn}
