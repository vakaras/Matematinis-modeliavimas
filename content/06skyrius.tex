\Chapter{Aukštesnės eilės paprastos diferencialinės lygtys}

\section{Paprasčiausios diferencialinės lygtys, kurių eilę galima
sumažinti}

\cite[90]{konspektas}

\section{Tiesinės homogeninės antros eilės lygtys}

\cite[93]{konspektas}

\begin{defn}[Tiesinis diferencialinis reiškinys]
  Tarkime, funkcijos $a_{1}, a_{2}$ yra tolydžios intervale $(a, b)$ ir
  \begin{equation*}
    L(y) = y'' + a_{1}(x)y' + a_{2}(x)y.
  \end{equation*}
  Tada reiškinys $L(y)$ yra vadinamas antros eilės tiesiniu
  diferencialiniu reiškiniu, o operatorius $L$ antros eilės tiesiniu
  diferencialiniu operatoriumi. Jo apibrėžimo sritis yra funkcijų
  erdvė $C^{2}(a, b)$.
\end{defn}

$L(y)$ savybės:
\begin{enumerate}
  \item $L(\lambda \varphi) = \lambda L(\varphi),$
    $\forall \varphi \in C^{2}(a, b), \lambda \in \RSET$;
  \item $L(\varphi + \psi) = L(\varphi) + L(\psi),$
    $\forall \varphi, \psi \in C^{2}(a, b)$.
\end{enumerate}

Nagrinėsime tiesinę homogeninę antros eilės diferencialinę lygtį:
\begin{equation*}
  L(y) = 0.
\end{equation*}

\begin{align*}
  L(\varphi)
  &= L(c_{1}\varphi_{1} + c_{2}\varphi_{2}) \\
  &= L(c_{1}\varphi_{1}) + L(c_{2}\varphi_{2}) \\
  &= c_{1}L(\varphi_{1}) + c_{2}L(\varphi_{2}) \\
  &= c_{1} \cdot 0 + c_{2} \cdot 0 \\
  &= 0 \\
\end{align*}

\begin{defn}[%
  Funkcijų sistemos $\varphi_{1}, \varphi_{2}$ Vronskio determinantas]
  Tegu $\varphi_{1}, \varphi_{2}$ yra diferencijuojamos intervale
  $(a, b)$ funkcijos. Iš šių funkcijų ir jų išvestinių
  sudarytas determinantas vadinamas funkcijų sistemos
  $\varphi_{1}, \varphi_{2}$ Vronskio determinantas:
  \begin{equation*}
    W(x) =%
    \begin{vmatrix}
      \varphi_{1}(x) & \varphi_{2}(x) \\
      \varphi'_{1}(x) & \varphi'_{2}(x) \\
    \end{vmatrix}
  \end{equation*}
\end{defn}

\section{Konstantų varijavimo metodas}

\cite[97]{konspektas}

\begin{align*}
  \left\{%
  \begin{aligned}
    v(x) &= c_{1}(x) \varphi_{1}(x) + c_{2}(x) \varphi_{2}(x) \\
    v'(x) &= c_{1}(x) \varphi'_{1}(x) + c_{2}(x) \varphi'_{2}(x) + %
      \underbrace{c'_{1}(x) \varphi_{1}(x) + c'_{2}(x) \varphi_{2}(x)}_{%
      0} \\
    v''(x) &= c_{1}(x) \varphi''_{1}(x) + c_{2}(x) \varphi''_{2}(x) + %
      c'_{1}(x) \varphi'_{1}(x) + c'_{2}(x) \varphi'_{2}(x) \\
  \end{aligned}
  \right. \\
  +\left\{%
  \begin{aligned}
    a_{2}(x)v(x) &=%
      a_{2}(x)c_{1}(x) \varphi_{1}(x) +%
      a_{2}(x)c_{2}(x) \varphi_{2}(x) \\
    a_{1}(x)v'(x) &=%
      a_{1}(x)c_{1}(x) \varphi'_{1}(x) +%
      a_{1}(x)c_{2}(x) \varphi'_{2}(x) \\
    v''(x) &= %
      c_{1}(x) \varphi''_{1}(x) + c_{2}(x) \varphi''_{2}(x) + %
      c'_{1}(x) \varphi'_{1}(x) + c'_{2}(x) \varphi'_{2}(x) \\
  \end{aligned}
  \right. \\
  \underbrace{v''(x) + a_{1}(x)v'(x) + a_{2}(x)v(x)}_{L(y)} =%
    a_{2}(x)c_{1}(x) \varphi_{1}(x) + \\
    a_{2}(x)c_{2}(x) \varphi_{2}(x) + \\
    a_{1}(x)c_{1}(x) \varphi'_{1}(x) + \\
    a_{1}(x)c_{2}(x) \varphi'_{2}(x) + \\
    c_{1}(x) \varphi''_{1}(x) + c_{2}(x) \varphi''_{2}(x) + \\
    c'_{1}(x) \varphi'_{1}(x) + c'_{2}(x) \varphi'_{2}(x) \\
  L(y) = %
    \underbrace{
      a_{2}(x)c_{1}(x) \varphi_{1}(x) + 
      a_{1}(x)c_{1}(x) \varphi'_{1}(x) + 
      c_{1}(x) \varphi''_{1}(x)
      }_{c_{1}(x)L(\varphi_{1}(x)) = 0} + \\
    \underbrace{
      a_{2}(x)c_{2}(x) \varphi_{2}(x) +
      a_{1}(x)c_{2}(x) \varphi'_{2}(x) +
      c_{2}(x) \varphi''_{2}(x)
      }_{c_{2}(x)L(\varphi_{2}(x)) = 0} + \\
    c'_{1}(x) \varphi'_{1}(x) +
    c'_{2}(x) \varphi'_{2}(x) \\
  L(y) = c'_{1}(x) \varphi'_{1}(x) + c'_{2}(x) \varphi'_{2}(x) \\
\end{align*}
Funkcija $L(y)$ tenkins lygtį, jei:
\begin{equation*}
  c'_{1}(x) \varphi'_{1}(x) + c'_{2}(x) \varphi'_{2}(x) = q(x).
\end{equation*}

Taigi reikalavimai konstantoms:
\begin{equation*}
  \left\{%
  \begin{aligned}
    c'_{1}(x)\varphi_{1}(x) + c'_{2}(x)\varphi_{2}(x) &= 0 \\
    c'_{1}(x)\varphi'_{1}(x) + c'_{2}(x)\varphi'_{2}(x) &= q(x) \\
  \end{aligned}.
  \right.
\end{equation*}
Išsprendę gauname:
\begin{align*}
  c'_{1}(x)
  &= \frac{
  \begin{vmatrix}
    0 & \varphi_{2}(x) \\
    q(x) & \varphi'_{2}(x) \\
  \end{vmatrix}
  }{
  \begin{vmatrix}
    \varphi_{1}(x) & \varphi_{2}(x) \\
    \varphi'_{1}(x) & \varphi'_{2}(x) \\
  \end{vmatrix}
  }
  = \frac{W_{1}(x)}{W(x)} \\
  c'_{2}(x)
  &= \frac{
  \begin{vmatrix}
    \varphi_{1}(x) & 0 \\
    \varphi'_{1}(x) & q(x) \\
  \end{vmatrix}
  }{
  \begin{vmatrix}
    \varphi_{1}(x) & \varphi_{2}(x) \\
    \varphi'_{1}(x) & \varphi'_{2}(x) \\
  \end{vmatrix}
  }
  = \frac{W_{2}(x)}{W(x)} \\
\end{align*}

\section{Tiesinės antros eilės lygtys su pastoviais realiais
koeficientais}

\section{Stygos svyravimų lygtis}

\cite[41]{konspektas}

\section{Membranos svyravimas}

\cite[44]{konspektas}

\section{Šilumos laidumas ir dujų difuzija}

\cite[47]{konspektas}
